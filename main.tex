\documentclass[11pt,twoside,a4paper]{scrartcl}

\usepackage{etseib}

% Definir l'idioma. Per defecte en anglès. Per català o castellà descomentar l'opció adecuada i asssegurar-se de tenir instal·lat el paquet texlive-lang-spanish
\usepackage[catalan]{babel}
%\usepackage[spanish]{babel}


\tfmTitle{Hola, això és un títol}
\tfmAuthor{Jordi Soler Busquets}{Master Degree in Automatic Control and Robotics}


\begin{document}
%    PORTADA
\maketitle % Print the title
\logoupc
\thispagestyle{empty}
\newpage
\pagenumbering{Roman}
\blank
\pautes

% 	 RESUM
\section*{Resum}
\addcontentsline{toc}{section}{\protect\numberline{}Resum}%

Això és un resum, molt resumit

\newpage

%    ÍNDEX
%\thispagestyle{empty}
\tableofcontents
\newpage
\pagenumbering{arabic}


% L'existència d'un glossari no és obligatòria, però sí aconsellable en el cas que apareguin signes, abreviatures, acrònims...
% Ha d'establir breument l'abast i els objectius del treball que es descriuen en el document. 

\section{Introducció}
Lorem ipsum dolor sit amet, consectetur adipiscing elit, sed do eiusmod tempor incididunt ut labore et dolore magna aliqua. Ut enim ad minim veniam, quis nostrud exercitation ullamco laboris nisi ut aliquip ex ea commodo consequat. Duis aute irure dolor in reprehenderit in voluptate velit esse cillum dolore eu fugiat nulla pariatur. Excepteur sint occaecat cupidatat non proident, sunt in culpa qui officia deserunt mollit anim id est laborum.

Sed ut perspiciatis unde omnis iste natus error sit voluptatem accusantium doloremque laudantium, totam rem aperiam, eaque ipsa quae ab illo inventore veritatis et quasi architecto beatae vitae dicta sunt explicabo. Nemo enim ipsam voluptatem quia voluptas sit aspernatur aut odit aut fugit, sed quia consequuntur magni dolores eos qui ratione voluptatem sequi nesciunt. Neque porro quisquam est, qui dolorem ipsum quia dolor sit amet, consectetur, adipisci velit, sed quia non numquam eius modi tempora incidunt ut labore et dolore magnam aliquam quaerat voluptatem. Ut enim ad minima veniam, quis nostrum exercitationem ullam corporis suscipit laboriosam, nisi ut aliquid ex ea commodi consequatur? Quis autem vel eum iure reprehenderit qui in ea voluptate velit esse quam nihil molestiae consequatur, vel illum qui dolorem eum fugiat quo voluptas nulla pariatur?


\newpage
\section{Una secció}
Això és una secció.

\subsection{Una subsecció}
Subsecció amb una referència a \cite{bib1}
\input{tex/nucli/seccio2.tex}
% Ha d'establir el cost del treball realitzat. 
% S'inclourà l'impacte medi ambiental, en el cas que es consideri oportú. 

\newpage
% Han de ser un reflex clar i ordenat de les deduccions a les quals s'ha arribat un cop realitzat el treball. 

\section{Conclusions}
Aquestes són les conclusions:

\begin{itemize}
\item Una
\item Dues
\item Tres
\end{itemize}


\newpage
% Poden incloure's o no agraïments relatius als ajuts en la realització del treball.

\section{Agraïments}
Gràcies.

%    BIBLIOGRAFIA
\newpage

% Segons les pautes de la ETSEIB l'estil de la bibliografia hauria de ser el descrit en la ISO 690:1987. Aquest estil (acm) és el més semblant, tot i que se'n podria crear un de nou...
\bibliographystyle{acm}
\bibliography{tex/bibliografia}


\end{document}